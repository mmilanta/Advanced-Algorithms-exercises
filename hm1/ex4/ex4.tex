\documentclass[11pt]{article}

\usepackage[a4paper,margin=0.9in]{geometry}
\usepackage{amsfonts}
\usepackage{amsmath}
\usepackage{amsthm}
\usepackage{cleveref}
\usepackage{cancel}
\usepackage{ifthen}
\usepackage{rotating}


\begin{document}
\author{Marco Milanta}
\title{Graded Homework 1, exercise 4}
\maketitle
% actually edit those files and create new sections if you need more. 

%% Special characters for number sets.
\newcommand{\N}{\mathbb{N}}


\newcommand{\R}{\mathbb{R}}
\newcommand{\E}{\mathbb{E}}

%% Vectors and sets.
\newcommand{\set}[1]{\mathcal #1}
\newcommand{\mat}[1]{\mathbf #1}
\renewcommand{\vec}[1]{\mathbf #1}
\newcommand{\x}{\mathbf{x}}
\newcommand{\w}{\mathbf{w}}
\newcommand{\X}{\mathcal{X}}
\newcommand{\D}{\mathcal{D}}
\newcommand{\ke}{\mathbf{k}}
\newcommand{\K}{\mathbf{K}}
\newcommand{\A}{\mathbf{A}}
\newcommand{\Y}{\mathbf{Y}}
\newcommand{\Si}{\Sigma}
%\newcommand{\Pr}{\mathbb{P}}

%% Bold greek letters.
\newcommand{\lm}{\boldsymbol{\lambda}}
\newcommand{\tht}{\boldsymbol{\theta}}
\newcommand{\ph}{\boldsymbol{\phi}}
\newcommand{\omg}{\boldsymbol{\omega}}

%% Theorems
\newtheorem{observation}{Observation}
\section*{Vertex Cover (15 points)}%\begin{enumerate}
    %\item 
    Consider the linear program that is used to 2-approximate the weighted vertex cover problem, that is we minimize $\sum_{u \in V(G)} w_u x_u$ given $x_u + x_v  \geq 1$ for every edge $uv$ and $x_u \ge 0$ for each node $u$. 
    %You can assume that there is a polynomial time algorithm that finds an optimum solution to this linear program. 
    Prove that there is a polynomial time algorithm that finds an optimum solution to this linear program such that it has the additional property that for every node $u$ we have $x_u \in \{0, 1/2, 1 \}$. 
    %\item Give a 3/2 approximation algorithm for vertex cover in planar graphs. You can assume the existence of the procedure for part 1 of this problem, even if you did not solve it. 
    %You can also assume that you have a polynomial time algorithm for 4-coloring planar graphs.
    %\item There is an analogous LP for hypergraph vertex cover. Is it the case that if the hypergraph is $k$-uniform, there is an optimum solution with all values being equal to $i/k$?
%\end{enumerate} \section*{Solution}
    \section*{Solution}
    We start by taking the valid solution that we get from linear programming, and then we transform it in a way that we get $x_u \in \{0,1/2,1\}$. To do so we create an iterative step that gets as input a valid solution, where some $x_u$ are in $\{0,1/2,1\}$ and some are not; and outputs a solution where all the $x_u$ that were inside $\{0,1/2,1\}$ in the input, will remain inside in the output; and at least one of those $x_u$ which were not inside $\{0,1/2,1\}$ in the input, will be inside $\{0,1/2,1\}$ in the output. Furthermore, the output solution will have a cost no greater than the input one. \\\\
    If we can devise an algorithm to run such step in polynomial time, then we are done. This is because we can iterate the algorithm $n$ times, where $n$ is the number of vertices. At each iteration one more $x_u$ will be inside $\{0,1/2,1\}$. Since we do $n$ iterations, in the end, we get a solution with the same cost as the original one (minimal), and with all the $x_u\in\{0,1/2,1\}$.
    \paragraph*{Iteration step:} The iteration step gets as input $x_u,\forall u \in V$. Let's define
    \begin{align*}
        \mathcal{V}_+ &= \left\{ u \in V \mid 1/2 < x_u < 1 \right\}\\
        \mathcal{V}_- &= \left\{ u \in V \mid 0 < x_u < 1/2 \right\}
    \end{align*}
    Here we assume that either $\mathcal{V}_+$ or $\mathcal{V}_-$ are not empty. If they are empty, then we are in a trivial situation. This is because $x_u\notin\{0,1/2,1\} \Rightarrow x_u > 1$. But if that is the case we can just update $\hat x_u = 1 \forall
    u$ such that $x_u > 1$. This will still yield a valid solution and decrease the cost (notice that this cannot happen in practice, because this would imply that our LP solution is not optimal). \\\\
    Now we create a new solution:
    \begin{equation*}
        \hat x_u(\delta)= \begin{cases}
            x_u - \delta & u \in \mathcal{V}_+\\
            x_u + \delta & u \in \mathcal{V}_-\\
            x_u & \text{otherwise}
        \end{cases}
    \end{equation*}
    We investigate for which value of $\delta$, $\hat x_u(\delta)$ is still a solution. The conditions to be respected are:
    \begin{enumerate}
        \item\label{c:1} $\hat x_u(\delta) \geq 0$: therefore $\delta \leq \min_{u\in \mathcal{V}_+}x_u, \delta \geq - \min_{u\in \mathcal{V}_-}x_u$
        \item\label{c:2} Given an edge $uv$, assuming $u \in \mathcal{V}_+$ and $v \in \mathcal{V}_-$, $\hat x_u(\delta)+\hat x_v(\delta) \geq 1$. This is always true since
        \begin{equation*}
            \hat x_u(\delta)+\hat x_v(\delta) = x_u - \cancel{\delta} + x_v + \cancel{\delta} = x_u + x_v \geq 1.
        \end{equation*}
        The same holds if $u \in \mathcal{V}_-$ and $v \in \mathcal{V}_+$. 
        \item\label{c:3} Given an edge $uv$, assuming $u \in \mathcal{V}_+$ and $x_v = 1/2$. Then we have 
        \begin{align*}
            \hat x_u(\delta)+\hat x_v(\delta) = x_u - \delta + 1/2 \geq 1
        \end{align*}
        Therefore we have that $\delta \leq x_u-1/2$.
    \end{enumerate}
    If we think about, all the other scenarios are not problematic. If we have one of the vertices is $0$, then the other one must be at least $1$, therefore it will not be inside $\mathcal{V}_+$ or $\mathcal{V}_-$, and it will not be modified. If we have one of the vertices is $1$, then the only condition for the other vertex is that it cannot drop below $0$, but this has already been encoded in condition \ref{c:1}. Finally, if one of vertices is in $\mathcal{V}_-$, then the other one must either be in $\mathcal{V}_+$ (condition \ref{c:2}) or it's $\geq 1$. It cannot be less than one, otherwise the input solution would not be valid.
    \\\\
    Now we notice that $\delta$ can be taken to be $>0$ or $<0$. This is because:
    \begin{itemize}
        \item $\min_{u\in \mathcal{V}_+}x_u \geq .5 > 0$
        \item $x_u-1/2 > 0$ since $x_u \in \mathcal{V}_+,$ and $x_u>1/2$
        \item $-\min_{u\in \mathcal{V}_-}x_u < 0$
    \end{itemize}
    From this we can look at
    \begin{equation*}
        D := \partial_\delta \left(\sum_{u\in V} w_ux_u\right) 
    \end{equation*}
    But, since the solution is optimal, we know that $D=0$. If, by contradiction, $D\neq 0$, we would have that picking a small $\delta$, either positive or negative, because the cost is linear in $\delta$, we improves the result. But this is impossible by optimality. Hence $D = 0$, hence any $\delta$ which is valid, yields the same cost as the original problem.\\\\
    Now, we take $\hat \delta$ to be the largest $\delta$ that is still valid. Since it's valid, of course $\hat x_u (\hat\delta)$ is still a solution. This would be our output.\\\\
    The last thing to show is that $\hat x_u (\hat\delta)$ has one more variable in $\{0,1/2,1\}$. But this is not hard: assume that the dominating condition for $\hat \delta$ is condition \ref{c:1}, in this scenario, there would be $x_u$ that goes to $0$. In the case where the dominating condition is condition \ref{c:3}, then we have that $\hat\delta = x_u -1/2$. This would yield that:
    \begin{equation*}
        \hat x_u(\hat\delta) = x_u-(x_u-1/2) = 1/2.
    \end{equation*}
    And therefore, we still have one more variable in $\{0,1/2,1\}$. Finally, it's easy to see that all the variables which where already in $\{0,1/2,1\}$ have not been changed.
\end{document}