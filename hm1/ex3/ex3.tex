\documentclass[11pt]{article}

\usepackage[a4paper,margin=0.9in]{geometry}
\usepackage{amsfonts}
\usepackage{amsmath}
\usepackage{amsthm}
\usepackage{cleveref}
\usepackage{ifthen}
\usepackage{rotating}


\begin{document}
\author{Marco Milanta}
\title{Graded Homework 1, exercise 3}
\maketitle
% actually edit those files and create new sections if you need more. 

%% Special characters for number sets.
\newcommand{\N}{\mathbb{N}}


\newcommand{\R}{\mathbb{R}}
\newcommand{\E}{\mathbb{E}}

%% Vectors and sets.
\newcommand{\set}[1]{\mathcal #1}
\newcommand{\mat}[1]{\mathbf #1}
\renewcommand{\vec}[1]{\mathbf #1}
\newcommand{\x}{\mathbf{x}}
\newcommand{\w}{\mathbf{w}}
\newcommand{\X}{\mathcal{X}}
\newcommand{\D}{\mathcal{D}}
\newcommand{\ke}{\mathbf{k}}
\newcommand{\K}{\mathbf{K}}
\newcommand{\A}{\mathbf{A}}
\newcommand{\Y}{\mathbf{Y}}
\newcommand{\Si}{\Sigma}
%\newcommand{\Pr}{\mathbb{P}}

%% Bold greek letters.
\newcommand{\lm}{\boldsymbol{\lambda}}
\newcommand{\tht}{\boldsymbol{\theta}}
\newcommand{\ph}{\boldsymbol{\phi}}
\newcommand{\omg}{\boldsymbol{\omega}}

%% Theorems
\newtheorem{observation}{Observation}
\section*{Avid Traveler (15 points)}There are $n$ cities $\{1, \cdots, n \}$ in Europe,  and a set of $m$ local train tickets $\{T_1, \cdots, T_m\}$, where $T_i$ is a list of cities you can travel to using the $i$th ticket, and let $p_i \in \{1,\cdots,n^{100}\}$ to be the price of the $i$th ticket. We are guaranteed that each city $c_i$ is in at least $1$ local ticket, but is in at most $10$ local tickets. There are also $10$ super tickets $\{S_1, \cdots S_{10}\}$, each costs $s_i$. You want to buy a set of tickets such that you can travel to any city in Europe while minimizing the cost. Moreover, as a condition for buying super ticket, we are not allowed to have a city simultaneously in a local ticket and a super ticket we bought (i.e. $T_i\cap S_j = \emptyset$ for any $T_i$ and $S_j$ we buy). Devise a polynomial-time approximation algorithm for this problem with a constant approximation guarantee, and for the smallest approximation factor that you can achieve.

\section*{Solution}
    \paragraph*{Fix the super tickets:} Suppose we are taking some super tikets 
    \begin{equation*}
        S_{i_1}, \dots, S_{i_k} \subseteq \{S_1,\dots,S_{10}\},\qquad k\leq 10. 
    \end{equation*}
    Now we want to cover the leftover cities with local train tickets. The leftover cities will be
    \begin{equation*}
        C_{i_1,\dots, i_k} := \{1,\dots,n\} \setminus \bigcup_{j=1}^kS_{i_j}
    \end{equation*}
    Furthermore, we consider the set of local tickets that we can still buy considering the condition that there can not be a city both in a local ticket and in a super ticket:
    \begin{equation*}
        \mathcal{T}_{i_1,\dots, i_k} := \left\{T\in \{T_1,\dots,T_m\}\mid T\cap S_{i_j} = \emptyset, j=1,\dots,k \right\}
    \end{equation*}
    Now we can ask ourselves what is the smallest set covering of sets in $\mathcal{T}_{i_1,\dots, i_k}$ to cover $C_{i_1,\dots, i_k}$. To solve this problem, we use a greedy algorithm based on the assumption that each city in $C_{i_1,\dots, i_k}$ will appear in at most $10$ trains. We have theorem 4.4 from lecture notes that guarantees us a polynomial approximated algorithm that finds a solution $\mathcal{T}'\subseteq\mathcal{T}_{i_1,\dots, i_k}$ that minimizes:
    \begin{equation*}
        c(\mathcal{A}_{i_1,\dots, i_k}) = \sum_{t\in \mathcal{T}'} p_t
    \end{equation*}
    where $\mathcal{T}'$ is a set cover of $C_{i_1,\dots, i_k}$. The approximation will only have a factor of $10$
    \paragraph*{For all possible super tickets combinations:} The idea is to iterate the previous point for all possible subset of $\{S_1,\dots,S_{10}\}$. This yields $2^{10}=1024$ different iterations. Even though this number is big, it doesn't scale up: the number of super tickets is not a parameter, but it is fixed to be $10$.\\\\
    Finally, we have $1024$ different solutions, and we pick the best one considering also the cost of the super tickets
    \begin{equation*}
        c(\mathcal{A}) = \min_{\{i_1,\dots,i_k\} \in 2^{1:k}} \left(\sum_{j = i_1,\dots,i_k}s_{j} + c(\mathcal{A}_{i_1,\dots, i_k})\right).
    \end{equation*}
    Note that we cannot guarantee that each combination of $S_{i_1}, \dots, S_{i_k}$ yields a valid solution, but we can guarantee that for at least one combination it happens: we don't take any super tickets. Therefore, the minimum is still well-defined (we just take $c(\mathcal{A}_{i_1,\dots, i_k}) = \infty$ if $\{S_{i_1},\dots,S_{i_k}\}$ doesn't allow any solution). Assume that the optimal solution picks $\mathcal{S}_{OPT}\subseteq \{S_1,\dots,S_m\}$ as super tickets. Then we know that
    \begin{equation*}
        c(\mathcal{A}) \leq \sum_{j\in \mathcal{S}_{OPT}}s_{j} +c(\mathcal{S}_{OPT}) \leq 10 c(OPT).
    \end{equation*}
    Finally, we can conclude that we have found a $10$-approximation algorithm. $10$ was the best we could do, and indeed, we suspect that it is impossible to do better than this. 
    \end{document}