\documentclass[11pt]{article}

\usepackage[a4paper,margin=0.9in]{geometry}
\usepackage{amsfonts}
\usepackage{amsmath}
\usepackage{amsthm}
\usepackage{cleveref}
\usepackage{ifthen}
\usepackage{rotating}


\begin{document}
\author{Marco Milanta}
\title{Graded Homework 1, exercise 6}
\maketitle
% actually edit those files and create new sections if you need more. 

%% Special characters for number sets.
\newcommand{\N}{\mathbb{N}}


\newcommand{\R}{\mathbb{R}}
\newcommand{\E}{\mathbb{E}}

%% Vectors and sets.
\newcommand{\set}[1]{\mathcal #1}
\newcommand{\mat}[1]{\mathbf #1}
\renewcommand{\vec}[1]{\mathbf #1}
\newcommand{\x}{\mathbf{x}}
\newcommand{\w}{\mathbf{w}}
\newcommand{\X}{\mathcal{X}}
\newcommand{\D}{\mathcal{D}}
\newcommand{\ke}{\mathbf{k}}
\newcommand{\K}{\mathbf{K}}
\newcommand{\A}{\mathbf{A}}
\newcommand{\Y}{\mathbf{Y}}
\newcommand{\Si}{\Sigma}
%\newcommand{\Pr}{\mathbb{P}}

%% Bold greek letters.
\newcommand{\lm}{\boldsymbol{\lambda}}
\newcommand{\tht}{\boldsymbol{\theta}}
\newcommand{\ph}{\boldsymbol{\phi}}
\newcommand{\omg}{\boldsymbol{\omega}}

%% Theorems
\newtheorem{observation}{Observation}
\section*{Fast Streaming Algorithm for MST (20 points)}

In the exercises of week 5, we discussed a streaming algorithm for computing a minimum spanning tree (MST) in $O(\log^2{n})$ passes. The goal of this question is to show a faster algorithm. Recall that in a $k$-pass streaming algorithm the algorithm is allowed to have $k$ passes over the input graph. 

Design a streaming algorithm for computing the MST of the graph, using $\tilde{O}(n)$ total memory and $O(\log{n} \log{\log{n}})$ passes. The algorithm can use randomization and should work with high probability. You can assume that the weights of the edges in the input graph are non-negative integers in $\{1, 2, \dots, n^{10}\}$.

\paragraph{Hint.} Note that if you want to compute outgoing edges from a small number of connected components, you can sample a large number of edges adjacent to each connected component without violating the memory constraints of the algorithm. Show how to exploit this.

\section*{Solution}

\end{document}