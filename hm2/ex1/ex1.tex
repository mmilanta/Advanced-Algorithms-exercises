\documentclass[11pt]{article}

\usepackage[a4paper,margin=0.9in]{geometry}
\usepackage{amsfonts}
\usepackage{amsmath}
\usepackage{amsthm}
\usepackage{cleveref}
\usepackage{ifthen}
\usepackage{rotating}
\usepackage{cancel}


\begin{document}
\author{Marco Milanta}
\title{Graded Homework 2, exercise 1}
\maketitle
% actually edit those files and create new sections if you need more. 

%% Special characters for number sets.
\newcommand{\N}{\mathbb{N}}


\newcommand{\R}{\mathbb{R}}
\newcommand{\E}{\mathbb{E}}

%% Vectors and sets.
\newcommand{\set}[1]{\mathcal #1}
\newcommand{\mat}[1]{\mathbf #1}
\renewcommand{\vec}[1]{\mathbf #1}
\newcommand{\x}{\mathbf{x}}
\newcommand{\w}{\mathbf{w}}
\newcommand{\X}{\mathcal{X}}
\newcommand{\D}{\mathcal{D}}
\newcommand{\ke}{\mathbf{k}}
\newcommand{\K}{\mathbf{K}}
\newcommand{\A}{\mathbf{A}}
\newcommand{\Y}{\mathbf{Y}}
\newcommand{\Si}{\Sigma}
%\newcommand{\Pr}{\mathbb{P}}

%% Bold greek letters.
\newcommand{\lm}{\boldsymbol{\lambda}}
\newcommand{\tht}{\boldsymbol{\theta}}
\newcommand{\ph}{\boldsymbol{\phi}}
\newcommand{\omg}{\boldsymbol{\omega}}

%% Theorems
\newtheorem{observation}{Observation}
\section*{Randomized Ski Rental and Yao's Principle (25 points)}

Recall that the in the lecture we demonstrated a deterministic $2$-competitive algorithm for Ski Rental and noted that no $(2 - \varepsilon$)-competitive deterministic algorithm exists for any $\varepsilon > 0$.
\begin{enumerate}
\item Show that, given a known input distribution (the algorithm knows in advance the probability of the season lasting exactly i days), there exists an at most $1.99$-competitive deterministic algorithm. Hint: the standard algorithm shown in class is 1-competitive if the season length is of length at most $B$.
\item{} [Nothing to submit] Convince yourself that Yao's principle is equivalent to proving ``any randomized algorithm running on an unknown input distributions cannot be $\alpha$-competitive'' by proving ``any deterministic algorithm running on a known input distribution cannot be $\alpha$-competitive''. Conclude that, due to subtask 1, it is impossible to show a lower bound of $2$ for competitiveness of randomized Ski Rental algorithms using Yao's principle. 
\item Yao's principle is tight under very mild technical conditions. In particular, Yao's principle is tight for Ski Rental. Demonstrate this by designing a $1.99$-competitive randomized algorithm for Ski Rental (the algorithm does not know the probabilities in advance). Note: the competitive ratio in this subtask does not need to match nor reference the one in subtask 1). Hint: since the algorithm does not know the input distribution, the worst-case input can be assumed to be deterministic.
\end{enumerate}
\section*{Solution}
To make the solution easier to reed we will follow this notation:
\begin{itemize}
    \item $\sigma \in \mathbb{N}$ is the number of days until the end of the season. 
    \item $\Alg_t$ is the algorithm that for the first $t$ days rent the skies, and then it buys them. 
    \item $c(\Alg, \sigma)$ is the cost of the algorithm $\Alg$ over the input $\sigma$. 
    \item $OPT(\sigma) = \min\{\sigma, B\}$ is the cost of the optima offline algorithm. 
\end{itemize}
\begin{enumerate}
    \item First, we notice that the algorithm $\Alg_{B-1}$ (note that this is just $\Alg_t$ with $t = B-1$) yields a competitive ratio of $\frac{2B-1}{B}$. If $\sigma < B$,
    \begin{equation*}
        c(\Alg_{B-1}, \sigma) = \sigma = OPT(\sigma).
    \end{equation*}
    On the other hand, if $\sigma \geq B$, 
    \begin{equation*}
        c(\Alg_{B-1}, \sigma) = B+B-1 = \frac{2B-1}{B}B=\frac{2B-1}{B}OPT(\sigma).
    \end{equation*}
    If $B<100$, then $\frac{2B-1}{B} < 1.99$. And $\Alg_{B-1}$ is the desired algorithm. Now we focus on the case where $B > 100$. Let $\mathcal P$ be the distribution over the inputs. We define the following elements to work better with $\mathcal{P}$ distribution:
    \begin{align*}
        \tilde B &:= \lceil0.8B \rceil\\
        p_1 &:= \Pr( \sigma <  \tilde B)\\
        p_2 &:= \Pr(\tilde B \leq \sigma <  B)\\
        p_3 &:= \Pr( \sigma \geq B)\\
        \Lambda_1 &:= \E[\sigma \mid \sigma <  \tilde B]\\
        \Lambda_2 &:= \E[\sigma \mid \tilde B \leq \sigma <  B].
    \end{align*}
    Our algorithm $\mathcal {B}$ will be either $\Alg_{\tilde B}$ or $\Alg_{B}$ depending on $\mathcal P$. In particular,
    \begin{align}
        \text{if:}&\qquad p_3 \leq \frac{1-\epsilon}{\epsilon} 0.8 p_2 &\text{ then: } \mathcal{B} = \Alg_{B} \label{c1}\\
        \text{else if:}&\qquad p_3 \geq \frac{0.8(1-\epsilon)-1}{\epsilon- 0.2 + 0.01}p_2 &\text{ then: } \mathcal{B} = \Alg_{\tilde{B}}\label{c2}
    \end{align} 
    We now present two claims that we will prove later:
    \begin{claim}[$\Alg_B$ completitive ratio]\label{cl1} As long as condition \ref{c1} is true, $\Alg_B$ is $2-\epsilon$ competitive.
    \end{claim}
    \begin{claim}[$\Alg_{\tilde B}$ completitive ratio]\label{cl2} As long as condition \ref{c2} is true, $\Alg_{\tilde B}$ is $2-\epsilon$ competitive.
    \end{claim}

    Notice that for $\epsilon = 0.1$ at least one of the two conditions must be true. The first condition becomes $p_3 \leq 7.2p_2$, the second one becomes $p_3 \geq \frac{28}{9}p_2 \simeq 3.11 p_2$. 

    This yields that, fixing $\epsilon = 0.1$, $\mathcal{R}$ will be a $2-0.1 = 1.9$ competitive algorithm for any possible input distribution $\mathcal P$. The only thing left is to prove the two claims. To help us with the proof of the claims, I first want to notice that we can write $\E_{\sigma \sim \mathcal{P}}[c(\Alg_{\tilde{B}}, \sigma)]$, $\E_{\sigma \sim \mathcal{P}}[c(\Alg_B, \sigma)]$ and $\E_{\sigma \sim \mathcal{P}}[OPT(\sigma)]$ as a function of $p_1,p_2,p_3,\Lambda_1,\Lambda_2$:
    \begin{align*}
        \E_{\sigma \sim \mathcal{P}}[c(\Alg_{B}, \sigma)] & = \E_{\sigma \sim \mathcal{P}}[c(\Alg_{B}, \sigma)\mid \sigma < \tilde B]\Pr(\sigma < \tilde B) \\ 
        &+ \E_{\sigma \sim \mathcal{P}}[c(\Alg_{B}, \sigma)\mid  \tilde B \leq \sigma < B]\Pr( \tilde B \leq \sigma < B) \\
        &+\E_{\sigma \sim \mathcal{P}}[c(\Alg_{B}, \sigma)\mid \sigma \geq B]\Pr( \sigma \geq B) \\
        &=\E_{\sigma \sim \mathcal{P}}[ \sigma \mid \sigma < B]\Pr(\sigma < B) \\ 
        &+\E_{\sigma \sim \mathcal{P}}[\sigma \mid B \leq \sigma < B]\Pr( B\leq \sigma < B) \\
        &+\E_{\sigma \sim \mathcal{P}}[2B \mid \sigma \geq B]\Pr( \sigma \geq B) \\
        &= \Lambda_1 p_1 + \Lambda_2 p_2 + 2Bp_3.
    \end{align*}
    \begin{align*}
        \E_{\sigma \sim \mathcal{P}}[c(\Alg_{\tilde B}, \sigma)] & = \E_{\sigma \sim \mathcal{P}}[c(\Alg_{\tilde B}, \sigma)\mid \sigma < \tilde B]\Pr(\sigma < \tilde B) \\ 
        &+ \E_{\sigma \sim \mathcal{P}}[c(\Alg_{\tilde B}, \sigma)\mid  \tilde B \leq \sigma < B]\Pr( \tilde B \leq \sigma < B) \\
        &+\E_{\sigma \sim \mathcal{P}}[c(\Alg_{\tilde B}, \sigma)\mid \sigma \geq B]\Pr( \sigma \geq B) \\
        &=\E_{\sigma \sim \mathcal{P}}[ \sigma \mid \sigma < B]\Pr(\sigma < B) \\ 
        &+\E_{B + \tilde B \sim \mathcal{P}}[\sigma \mid B \leq \sigma < B]\Pr( B\leq \sigma < B) \\
        &+\E_{\sigma \sim \mathcal{P}}[B + \tilde B \mid \sigma \geq B]\Pr( \sigma \geq B) \\
        &= \Lambda_1 p_1 + (B + \tilde B)p_2 + (B + \tilde B)p_3.
    \end{align*}
    \begin{align*}
        \E_{\sigma \sim \mathcal{P}}[OPT(\sigma)] & = \E_{\sigma \sim \mathcal{P}}[OPT(\sigma)\mid \sigma < \tilde B]\Pr(\sigma < \tilde B) \\ 
        &+ \E_{\sigma \sim \mathcal{P}}[OPT(\sigma)\mid  \tilde B \leq \sigma < B]\Pr( \tilde B \leq \sigma < B) \\
        &+\E_{\sigma \sim \mathcal{P}}[OPT(\sigma)\mid \sigma \geq B]\Pr( \sigma \geq B) \\
        &=\E_{\sigma \sim \mathcal{P}}[ \sigma \mid \sigma < B]\Pr(\sigma < B) \\ 
        &+\E_{\sigma \sim \mathcal{P}}[\sigma \mid B \leq \sigma < B]\Pr( B\leq \sigma < B) \\
        &+\E_{\sigma \sim \mathcal{P}}[B \mid \sigma \geq B]\Pr( \sigma \geq B) \\
        &= \Lambda_1 p_1 + \Lambda_2 p_2 + Bp_3.
    \end{align*}

    \begin{proof}[Proof of claim \ref{cl1}]
        We start by observing that
        \begin{align*}
            p_3 &\leq \frac{1-\epsilon}{\epsilon} 0.8 p_2 = \frac{1-\epsilon}{\epsilon} \frac{0.8B}{B} p_2 \leq \frac{1-\epsilon}{\epsilon} \frac{\Lambda_2}{B} p_2.
        \end{align*}
        Where the last inequality is true since
        \begin{equation*}
            \Lambda_2 = \E[\sigma \mid \tilde B \leq \sigma <  B] \geq \tilde B = \lceil0.8B \rceil \geq 0.8B.
        \end{equation*}
        Now, with some algebra, we can rewrite this as:
        \begin{align*}
            & p_3 \leq \frac{1-\epsilon}{\epsilon} \frac{\Lambda_2}{B}\\
            \Rightarrow & B \epsilon p_3 \leq (1-\epsilon)\Lambda_2
            p_2\\
            \Rightarrow & 0 \leq (1-\epsilon)\Lambda_2
            p_2 + Bp_3(1-\epsilon - 1)\\
            \Rightarrow & Bp_3 \leq (1-\epsilon)(\Lambda_2p_2 + Bp_3)\\
            \Rightarrow & 1-\epsilon \geq \frac{Bp_3}{\Lambda_2p_2 + Bp_3} \geq \frac{Bp_3}{\Lambda_2p_2 + Bp_3 + p_1 \Lambda_1}&\text{since $\Lambda_1 \geq 0$}\\
            \Rightarrow & 2-\epsilon \geq 1 + \frac{Bp_3}{\Lambda_2p_2 + Bp_3 + p_1 \Lambda_1} = \frac{\Lambda_2p_2 + 2Bp_3 + p_1 \Lambda_1}{\Lambda_2p_2 + Bp_3 + p_1 \Lambda_1}\\
            \Rightarrow & \underbrace{\Lambda_2p_2 + 2Bp_3 + p_1 \Lambda_1}_{\E_{\sigma \sim \mathcal{P}}[c(\Alg_B, \sigma)]} \leq (2-\epsilon)\underbrace{(\Lambda_2p_2 + Bp_3 + p_1 \Lambda_1)}_{\E_{\sigma \sim \mathcal{P}}[OPT(\sigma)]}\\
            \Rightarrow & \frac{\E_{\sigma \sim \mathcal{P}}[c(\Alg_B, \sigma)]}{\E_{\sigma \sim \mathcal{P}}[OPT(\sigma)]} \leq 2-\epsilon.
        \end{align*}
        The final statement is the definition of a $2-\epsilon$ competitive ratio algorithm. Hence, we have concluded the proof. 
    \end{proof}    
    \begin{proof}[Proof of claim \ref{cl2}]
        We start by observing that
        \begin{equation*}
            p_3 \geq \frac{0.8(1-\epsilon)-1}{\epsilon + 0.8 - 1 + 0.01}p_2 \geq \frac{0.8(1-\epsilon)-1}{\epsilon + 0.8 - 1 + \frac{1}{B}}p_2. 
        \end{equation*}
        Where this is true since we are assuming that $B > 100$. Now, with some algebra, we can rewrite this as:
        \begin{align*}
            & p_3 \geq \frac{0.8(1-\epsilon)-1}{\epsilon + 0.8 - 1 + \frac{1}{B}}p_2 \geq \frac{0.8(1-\epsilon)-1-\frac{1}{B}}{\underbrace{\epsilon + 0.8 - 1 + \frac{1}{B}}_{<0}}p_2 \\
            \Rightarrow & p_3\left(\epsilon + 0.8 - 1 + \frac{1}{B}\right) \leq p_2 \left(0.8(1-\epsilon)-1-\frac{1}{B}\right)\\
            \Rightarrow & -Bp_3(1-\epsilon) + 0.8Bp_3 + \cancel{\frac{1}{B}}\cancel{B}p_3 \leq 0.8(1-\epsilon)Bp_2 -(B+1)p_2\\
            \Rightarrow& (B+1)p_2 + (0.8B+1)p_3 \leq (1-\epsilon)(0.8Bp_2 + Bp_3)\\
            \Rightarrow& 1-\epsilon \geq \frac{(B+1)p_2 + (0.8B + 1)p_3}{0.8Bp_2 + Bp_3} \geq \frac{(B+1)p_2 + \tilde{B}p_3}{0.8Bp_2 + Bp_3} & \text{since }\tilde{B} = \lceil 0.8B\rceil \leq 0.8B + 1\\
            \Rightarrow& 2-\epsilon \geq 1 +  \frac{(B+1)p_2 + \tilde{B}p_3}{0.8Bp_2 + Bp_3} \geq 1 +  \frac{(B+1)p_2  + \tilde{B}p_3}{p_1\Lambda_1 + 0.8Bp_2 + Bp_3} & \text{since }\Lambda_1 \geq 0\\
            \Rightarrow& 2-\epsilon \geq \frac{p_1 \Lambda_1 + (B + 0.8B + 1)p_2 + (B+\tilde B)p_3}{p_1\Lambda_1 + 0.8Bp_2 + Bp_3}\\
            \Rightarrow& 2-\epsilon \geq \frac{p_1 \Lambda_1 + (B+\tilde B)p_2 + (B+\tilde B)p_3}{p_1\Lambda_1 + 0.8Bp_2 + Bp_3} &  \text{since }\tilde{B} = \lceil 0.8B\rceil \leq 0.8B + 1 \\
            \Rightarrow& 2-\epsilon \geq \frac{p_1 \Lambda_1 + (B+\tilde B)p_2 + (B+\tilde B)p_3}{p_1\Lambda_1 + \Lambda_2p_2 + Bp_3} &  \text{since }0.8B \leq \tilde{B} \leq \Lambda_2\\
            \Rightarrow& \underbrace{p_1 \Lambda_1 + (B+\tilde B)p_2 + (B+\tilde B)p_3}_{\E_{\sigma \sim \mathcal{P}}[c(\Alg_{\tilde B}, \sigma)]} \leq (2-\epsilon)\underbrace{(p_1\Lambda_1 + \Lambda_2p_2 + Bp_3)}_{\E_{\sigma \sim \mathcal{P}}[OPT(\sigma)]}\\
            \Rightarrow& \frac{\E_{\sigma \sim \mathcal{P}}[c(\Alg_{\tilde B}, \sigma)]}{\E_{\sigma \sim \mathcal{P}}[OPT(\sigma)]} \leq 2-\epsilon
        \end{align*}
        The final statement is the definition of a $2-\epsilon$ competitive ratio algorithm. Hence, we have concluded the proof.
    \end{proof}

    \item {}[Nothing to submit]
    \item In the case where $B > 100$, we propose the following randomized algorithm $\mathcal{R}
    $:
    \begin{equation*}
        \mathcal{R}(\sigma) = \begin{cases}
            \Alg_B(\sigma)& w.p. \;0.5\\
            \Alg_{\lceil 0.8B\rceil}(\sigma)& w.p. \;0.5\\
        \end{cases}
    \end{equation*}
    For simplicity, we call $\tilde B = \lceil 0.8B\rceil$.
    \paragraph{Case 1:} ($\sigma < \tilde B $) In this scenario, our algorithm yields the same output of the optimal algorithm. This yields a competitive ratio of $1$.
    \paragraph{Case 2:} ($\tilde B \leq \sigma < B$) In this scenario, the expected cost of our algorithm is:
    \begin{equation*}
        \E_{\mathcal R}[c(\mathcal{R},\sigma)] =\frac{1}{2}\underbrace{\sigma}_{\mathcal R = \Alg_B} + \frac{1}{2}\underbrace{(B+\tilde B)}_{\mathcal R = \Alg_{\tilde{B}}}  \leq \frac{1}{2}B + \frac{1}{2}(B + 0.8B + 1) = \frac{2.8}{2}B + \frac{1}{2} = 1.4B + 0.5 \stackrel{(i)}{\leq} 1.5B.
    \end{equation*}
    Where in $(i)$ we use that $B > 100$. In this same setting, the optimal algorithm yields a result of $\sigma\geq \tilde B \geq 0.8B$. Hence, we can notice that
    \begin{equation*}
        \E_{\mathcal R}[c(\mathcal{R},\sigma)] \leq 1.5B = 1.875 \times 0.8B\leq 1.875 \sigma = 1.875 \,OPT(\sigma).
    \end{equation*}
    Therefore, in this scenario, the competitive ratio is $1.875$
    \paragraph{Case 3:} ($\sigma \geq B$) In this scenario, the expected cost of our algorithm is:
    \begin{equation*}
        \E_{\mathcal R}[c(\mathcal{R},\sigma)] =\frac{1}{2}2B + \frac{1}{2}(B + \tilde B) = \frac{3}{2}B + \frac{1}{2}\tilde B \leq \frac{3}{2}B + \frac{1}{2}(0.8B +1) \leq 1.9B + 1 \stackrel{(i)}{\leq} 1.91B
    \end{equation*}
    Where in $(i)$ we use that $B > 100$. Whereas, the optimal algorithm yields a result of $B$. Hence, we can notice that
    \begin{equation*}
        \E_{\mathcal R}[c(\mathcal{R},\sigma)]  \leq 1.91B = 1.91\,OPT(\sigma).
    \end{equation*}
    Therefore, in this scenario, the competitive ratio is $1.91$.
    \paragraph*{Conclusion:}
    Since in the worst case input, the competitive ratio is $1.91$, we can write:
    \begin{equation*}
        \E_{\mathcal R}[c(\mathcal{R},\sigma)]  \leq \begin{cases}
            OPT(\sigma) & \text{in case }1\\
            1.875OPT(\sigma) & \text{in case }2\\
            1.91OPT(\sigma) & \text{in case }3
        \end{cases} \qquad\leq 1.91OPT(\sigma).
    \end{equation*}
    Hence, we have a competitive ratio of $1.91$ which is less than $1.99$. 
    
    Finally, if $B\leq 100$, then we use the deterministic algorithm $\Alg_{B-1}$ that achieves a competitive ratio of $\frac{2B-1}{B}$. Notice that if $B \leq 100$, then $\frac{2B-1}{B} \leq 1.99$.
    \paragraph*{Yao's principle:} Now that we have found this algorithm, we can use Yao's principle to "solve again" point 1 without explicating the algorithm.
\end{enumerate}
\end{document}