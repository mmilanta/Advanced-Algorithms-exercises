\documentclass[11pt]{article}

\usepackage[a4paper,margin=0.9in]{geometry}
\usepackage{amsfonts}
\usepackage{amsmath}
\usepackage{amsthm}
\usepackage{cleveref}
\usepackage{cancel}
\usepackage{ifthen}
\usepackage{rotating}


\begin{document}
\author{Marco Milanta}
\title{Graded Homework 1, exercise 4}
\maketitle
% actually edit those files and create new sections if you need more. 

%% Special characters for number sets.
\newcommand{\N}{\mathbb{N}}


\newcommand{\R}{\mathbb{R}}
\newcommand{\E}{\mathbb{E}}

%% Vectors and sets.
\newcommand{\set}[1]{\mathcal #1}
\newcommand{\mat}[1]{\mathbf #1}
\renewcommand{\vec}[1]{\mathbf #1}
\newcommand{\x}{\mathbf{x}}
\newcommand{\w}{\mathbf{w}}
\newcommand{\X}{\mathcal{X}}
\newcommand{\D}{\mathcal{D}}
\newcommand{\ke}{\mathbf{k}}
\newcommand{\K}{\mathbf{K}}
\newcommand{\A}{\mathbf{A}}
\newcommand{\Y}{\mathbf{Y}}
\newcommand{\Si}{\Sigma}
%\newcommand{\Pr}{\mathbb{P}}

%% Bold greek letters.
\newcommand{\lm}{\boldsymbol{\lambda}}
\newcommand{\tht}{\boldsymbol{\theta}}
\newcommand{\ph}{\boldsymbol{\phi}}
\newcommand{\omg}{\boldsymbol{\omega}}

%% Theorems
\newtheorem{observation}{Observation}
\section*{Number of Minimum Cuts (25 points)}
In the class, it was shown that the number of minimum cuts of any unweighted graph is at most $\binom{n}{2}$. Here we want to discuss two generalizations of this result.

\begin{enumerate}
    \item In the lecture we proved that the number of minimum cuts in any unweighted graph is at most $\binom{n}{2}$. Prove the same result for \emph{weighted graphs}. All the weights are positive.
    
    \item For a graph $G = (V, E)$, a subset of edges $E' \subseteq E$ is a $k$-cut if $G' = (V, E \setminus E')$ has at least $k$ connected components. Prove that the number of minimum $k$-cuts in any unweighted graph is at most $2^{O(k)}n^{2k-2}$.
\end{enumerate}

\section*{Solutions}
\begin{enumerate}
    \item Let $G=(V,E,w)$ be our weighted graph, $w:E\to \R^+$. Here $w(e)$ is the weight of the edge $e$. Let $n = \#V$. Let $M$ be the minimum cut size. 
    
    To prove that the number of minimum cut is at most $\binom{n}{2}$, we use a modified version of Karger's random contraction algorithm. When we sample edges, we don't do it uniformly, but the probability of sampling $e$ from $E'$ will be 
    \begin{equation*}
        \Pr(\text{samplig }e \text{ from }E') = \frac{w(e)}{\sum_{e\in E'}w(e)}.
    \end{equation*}
    We now prove the following lemma
    \begin{lemma}[Karger's efficency for weighted graphs]\label{l1} Given a minimum cut $\{S,V\setminus S\}$ of size $M$. The probability that the modified Karger's algorithm selects it is at least $\frac{1}{\binom{n}{2}}$
    \end{lemma}
    \begin{proof}
        Let $\tilde E$ be the set of edges that cut $\{S,V\setminus S\}$. We know that Krager's algorithm runs in no more than $n-2$ steps. We say that the algorithm fails if at any timestap it samples an element from $\tilde E$. Note that if the algorithm doesn't fail it outputs $\{S,V\setminus S\}$. To have more insight about why this is true one can look into Karger's algorithm on the lecture notes chapter 13.

        \paragraph*{Failing at step $i$:} We now look at the probability of failing at step $i$ assuming we didn't fail before.

        The probability of failing corresponds to the probability of sampling $e\in \tilde E$ from $E_i$. $E_i$ is the set of edges in the contracted graph, $V_i$ is the set of vertices which have not been contradiction. Note that $\# V_i = n - i + M$. Note that $\sum_{e \in E_i}w(e)\geq \frac{(n-i+1)M}{2}$. To show this we rewrite it as
        \begin{align*}
            \sum_{e \in E_i}w(e) &= \frac{1}{2} \sum_{v \in V_i}\sum_{e \text{ adj. to }v} w(e) \\
            & \geq  \frac{1}{2}\sum_{v \in V_i}M \\
            & = \frac{1}{2}\#V_i M = \frac{1}{2}(n-i+1)M
        \end{align*}
        Where in the first equality we rewrite the sum looking vertex by vertex at its outgoing edges. In this way, however, we count every vertex twice, therefore we add a $\frac{1}{2}$ factor to compensate. In the first line we use the fact that $\sum_{e \text{ adj. to }v} w(e)$ is just the size of the cut $\{\{e\},V\setminus \{e\}\}$, therefore it must be larger than $M$.

        Now, the probability of failing is 
        \begin{align*}
            \Pr\left(\text{sampling }e \in \tilde E \mid e \in E_i\right) &= \sum_{e \in \tilde E}\Pr\left(\text{sampling }e\mid e \in E_i\right)\\
            &= \sum_{e \in \tilde E}\frac{w(e)}{\sum_{e\in E'}w(e)} &\text{modified algoritm}\\
            & \leq \frac{\sum_{e \in \tilde E}w(e)}{\frac{(n-i+1)M}{2}} = \frac{M}{\frac{(n-i+1)M}{2}} = \frac{2}{n-i+1}. &\text{using the result above}
        \end{align*}
        \paragraph*{Never failing:} The probability of never failing is 
        \begin{align*}
            \Pr\left(\text{never failing}\right) &= \Pr\left(\text{not fail at step $n-2$}\mid \text{it didn't fail before $n-2$}\right) \cdots \Pr\left(\text{not fail at step $1$}\right)\\
            &=\prod_{i=1}^{n-2}\Pr\left(\text{not fail at step $i$}\mid \text{didn't fail before $i$}\right)\\
            & = \prod_{i=1}^{n-2}\left(1-\Pr\left(\text{fail at step $i$}\mid \text{didn't fail before $i$}\right)\right)\\
            & \geq \prod_{i=1}^{n-2}\left(1-\frac{2}{n-i+1}\right) \\
            & = \prod_{i=1}^{n-2}\left(\frac{n-i-1}{n-i+1}\right) \\
            & = \frac{\cancel{n-2}}{n}\frac{\cancel{n-3}}{n-1}\frac{\cancel{n-4}}{\cancel{n-2}}\cdots \frac{\cancel{3}}{\cancel{5}}\frac{2}{\cancel{4}}\frac{1}{\cancel{3}} = \frac{2}{n(n-1)} = \frac{1}{\binom{n}{2}}.
        \end{align*}
        Since the probability of not failing, is the probability to sample $\{S,V\setminus S\}$, we have concluded the proof.
    \end{proof}
    Finally, it's very easy to prove that there are no more than $\binom{n}{2}$ minimum cuts. By contradiction, assume that there were more than $\binom{n}{2}$ minimum cuts. Each one of them will have a probability of $\frac{1}{\binom{n}{2}}$ to be the output of Karger's algorithm. This means, that the total probability of Karger's to output one of the minimum cuts is 
    \begin{align*}
        \Pr(\text{sample a min. cut}) &= \sum_{i=1}^{\text{num. of min. cuts}>\binom{n}{2}}\Pr(\text{the $i$-th min. cut is selected}) \\
        &\geq  \sum_{i=1}^{\text{num. of min. cuts}>\binom{n}{2}} \frac{1}{\binom{n}{2}} > 1
    \end{align*}
    Where in the second line we used lemma \ref{l1}. Of course having a probability to be more than 1 is a contradiction. This implies that the total number of minimum cuts cannot be more than $\binom{n}{2}$. 
    \item The proof of this part follows a very similar strategy to the one seen before. First we need and important lemma.
    \begin{lemma}[Bound number of edges]\label{l2} Let $G = (V,E)$ be a graph (or multigraph) where, $n = \#V$ and the minimum $k-$cut has size $M$. Then
        \begin{equation*}
            \#E \geq \frac{M}{1-\left(1-\frac{k-1}{n}\right)\left(1-\frac{k-1}{n-1}\right)}.
        \end{equation*}
    \end{lemma}
    \begin{proof}
        
    \end{proof}
    Now, we propose to use the Karger's algorithm but stopping after $n-k$ iterations. The outputs will be a graph with $k$ nodes where each node correspond to a subset of $V$. The $k$ subsets will represent the selected $k$-cut. Now we can prove the key theorem
    \begin{theorem}[Karger's efficency for $k$-cuts]\label{th1} Given a graph $G=(V,E)$ where $\#V = n$. Let 
    \begin{equation*}
        C = \left\{A_1, A_2,\dots, A_{k-1}, V\setminus\left(\bigcup_{i=1}^{k-1}A_i\right)\right\}, \text{ where } A_i\cap A_j = \emptyset\; \forall i \neq j
    \end{equation*}  
        be a minimum $k$-cut. The probability that the early-stopped (after $n-k$ iterations) Karger's algorithm outputs $C$ is 
        \begin{equation*}
            \Pr\left(\text{Krager's outputs } C\right)\geq c\frac{(k-1)^{2(k-1)}}{(ne)^{2k-2}}
        \end{equation*}
    \end{theorem}
    \begin{proof}
        Let $\tilde E$ be the set of edges that cut $C$. Let $M = \#\tilde E$. We say that the algorithm fails if at some point it picks an edge from $\tilde E$. If it does so, the algorithm will not output $C$. If it does not, then it must output $C$. We divide the proof in two parts:
        \paragraph*{Failing at step $i$:}We now look at the probability of failing at step $i$ given that it has not failed before. Let $G_i = (V_i,E_i)$ the multigraph output of the previous steps. Note that the minimum cut of such graph must be $M$. Furthermore, note that $\#V_i=n-i+1$ (at step 1 $V_1 = V$, and then we take out one node per iteration).
        
        Thanks to lemma \ref{l2} we are guaranteed that
        \begin{equation*}
            \# E_i \geq \frac{M}{1-\left(1-\frac{k-1}{\#V_i}\right)\left(1-\frac{k-1}{\#V_i-1}\right)} = \frac{M}{1-\left(1-\frac{k-1}{n-i+1}\right)\left(1-\frac{k-1}{n-i}\right)}.
        \end{equation*}

        Let $\hat e$ be the edge sampled at the $i$-th step. Since we sample uniformly at random from $E_i$, we get that
        \begin{equation*}
            \Pr\left(\hat e \in \tilde E\right) \leq \frac{\#\tilde E}{\# E_i}=\frac{M}{\frac{M}{1-\left(1-\frac{k-1}{n-i+1}\right)\left(1-\frac{k-1}{n-i}\right)}} = 1-\left(1-\frac{k-1}{n-i+1}\right)\left(1-\frac{k-1}{n-i}\right).
        \end{equation*}
        \paragraph*{Never failing:} We now look at the probability that it never fails:
        \begin{align*}
            \Pr\left(\text{never failing}\right) &= \Pr\left(\text{not fail at step $n-2$}\mid \text{it didn't fail before $n-2$}\right) \cdots \Pr\left(\text{not fail at step $1$}\right)\\
            &=\prod_{i=1}^{n-k}\Pr\left(\text{not fail at step $i$}\mid \text{didn't fail before $i$}\right)\\
            & = \prod_{i=1}^{n-k}\left(1-\Pr\left(\text{fail at step $i$}\mid \text{didn't fail before $i$}\right)\right)\\
            & \geq \prod_{i=1}^{n-k}\left(\cancel{1}\cancel{-1}+\left(1-\frac{k-1}{n-i+1}\right)\left(1-\frac{k-1}{n-i}\right)\right) \\
            & = \prod_{i=1}^{n-k}\left(\frac{n-i-k+2}{n-i+1}\right)\left(\frac{n-i-k+1}{n-i}\right) \\
            & = \underbrace{\prod_{i=1}^{n-k}\left(\frac{n-i-k+2}{n-i+1}\right)}_{=A}\underbrace{\prod_{i=1}^{n-k}\left(\frac{n-i-k+1}{n-i}\right)}_{=B}
        \end{align*}
        Now we look at $A$ and $B$ independently. For $A$ we have:
        \begin{align*}
            A &= \prod_{i=1}^{n-k}\left(\frac{n-i-k+2}{n-i+1}\right)\\
            &= \frac{n-k+1}{n}\frac{n-k}{n-1}\frac{n-k-1}{n-2}\cdots \frac{n-2k+2}{n-k+1}\cdots \frac{k-1}{2k-4}\cdots \frac{4}{k+3}\frac{3}{k+2}\frac{2}{k+1}\\
            &= \frac{\cancel{n-k+1}}{n}\frac{\cancel{n-k}}{n-1}\frac{\cancel{n-k-1}}{n-2}\cdots \frac{\cancel{n-2k+2}}{\cancel{n-k+1}}\cdots \frac{\cancel{k+1}}{\cancel{2k}}\cdots \frac{4}{\cancel{k+3}}\frac{3}{\cancel{k+2}}\frac{2}{\cancel{k+1}}\\
            &= \frac{k!}{\frac{n!}{(n-k+1)!}} = \frac{k(k-1)!}{\frac{n!}{(n-k+1)!}} = \frac{k}{\frac{n!}{(n-(k-1))!(k-1)!}}= \frac{k}{\binom{n}{k-1}}.
        \end{align*}
        For $B$ we have:
        \begin{align*}
            B & = \prod_{i=1}^{n-k}\left(\frac{n-i-k+1}{n-i}\right)\\
            &=\frac{n-k}{n-1}\frac{n-k-1}{n-2}\frac{n-k-2}{n-3}\cdots\frac{n-2k+1}{n-k}\cdots \frac{k}{2k-1}\cdots\frac{3}{k+2}\frac{2}{k+1}\frac{1}{k}\\
            &= \frac{\cancel{n-k}}{n-1}\frac{\cancel{n-k-1}}{n-2}\frac{\cancel{n-k-2}}{n-3}\cdots\frac{\cancel{n-2k+1}}{\cancel{n-k}}\cdots \frac{\cancel{k}}{\cancel{2k-1}}\cdots\frac{3}{\cancel{k+2}}\frac{2}{\cancel{k+1}}\frac{1}{\cancel{k}}\\
            &=\frac{(k-1)!}{\frac{(n-1)!}{(n-k)!}} = \frac{1}{\frac{(n-1)!}{(n-1-(k-1))!(k-1)!}} = \frac{1}{\binom{n-1}{k-1}}
        \end{align*}
        Finally, we can bound the probability of never failing by using a well known bound on the binomial coefficient: $\binom{n}{k}\leq \frac{n^k}{k!} \leq \left(\frac{ne}{k}\right)^k$.
        \begin{align*}
            \Pr\left(\text{never failing}\right) &\geq AB = \frac{k}{\binom{n}{k-1}\binom{n-1}{k-1}}\\
            & \geq \frac{k}{\left(\frac{ne}{k-1}\right)^{k-1}\left(\frac{(n-1)e}{k-1}\right)^{k-1}}\\
            & \geq \frac{k}{\left(\frac{ne}{k-1}\right)^{2k-2}}\\
            & \geq \frac{k(k-1)^{2k-2}}{(ne)^{2k-2}}
        \end{align*}
        Since if the algorithm never fails, then it picks exactly $C$; we have concluded the proof. 
    \end{proof}
    Once we have theorem \ref{th1}, then the result is easy. Indeed, we cannot have more than
    \begin{equation*}
        \text{Number of minimum $k$-cuts} \leq \frac{(ne)^{2k-2}}{k(k-1)^{2k-2}}.
    \end{equation*}
    If, by contradiction we assume we had more, we get that the total probability of getting one of the minimum cuts is more than $1$, which is impossible (For a similar argument more in detail look at the last part of the previous point). Now, rearranging some terms we get:
    \begin{align*}
        \text{Number of minimum $k$-cuts} &\leq \frac{(ne)^{2k-2}}{k(k-1)^{2k-2}}\\
        & = e^{O(k)}n^{2k-2}\frac{1}{k(k-1)^{2k-2}} \leq 2^{O(k)}n^{2k-2}.
    \end{align*}
    Where $e^{O(k)} = 2^{O(k)\log_2e} = 2^{O(k)}$, and $\frac{1}{k(k-1)^{2k-2}} \in O(1)$. This concludes the exercise. As a final note, for values of $k$ which are not very small, $\frac{1}{k(k-1)^{2k-2}}$ is much smaller than $O(1)$, and a tighter bound could be obtained by not simplifying. 
\end{enumerate}
\end{document}